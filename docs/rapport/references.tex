\begin{thebibliography}{9}
    \bibitem{soniccoloru}
    Sonic Team (3 septembre 2021) \emph{Sonic Colours Ultimate}, SEGA.
    
    \bibitem{learnopengl} 
    Joey de Vries, \textit{LearnOpenGL}.  
    Disponible en ligne : \url{https://learnopengl.com/}.  
    LearnOpenGL est une ressource en ligne exhaustive dédiée à l'apprentissage de la programmation graphique avec OpenGL.  
    Le site couvre des sujets tels que la manipulation des shaders, la gestion des textures, les transformations 3D,  
    ainsi que des concepts avancés comme le rendu différé et les ombres en temps réel.  
    Consulté le 1 avril 2025.

    \bibitem{imanolfotia} 
    Imanol Fotia, \textit{Blog - Article 1}.  
    Disponible en ligne : \url{https://imanolfotia.com/blog/1}.  
    Ce blog aborde divers sujets liés à la programmation, à l'ingénierie logicielle et aux technologies modernes.  
    Il propose des analyses approfondies, des tutoriels techniques et des réflexions sur les bonnes pratiques en développement.  
    Consulté le 1 avril 2025.

    \bibitem{iehl_deferred_vs_forward} 
    Jean-Claude Iehl, \textit{Deferred Shading vs Forward Shading}.  
    Disponible en ligne : \url{https://perso.univ-lyon1.fr/jean-claude.iehl/Public/educ/M2PROIMA/2018/deferred_vs_forward.html}.  
    Ce document pédagogique explique les différences entre le rendu différé (Deferred Shading) et le rendu direct (Forward Shading).  
    Il couvre les principes fondamentaux de chaque technique, leurs avantages et inconvénients, ainsi que leurs applications  
    dans les moteurs de rendu modernes.  
    Consulté le 1 avril 2025.

    \bibitem{courreges_doom2016} 
    Adrian Courrèges, \textit{DOOM (2016) - Graphics Study}.  
    Disponible en ligne : \url{https://www.adriancourreges.com/blog/2016/09/09/doom-2016-graphics-study/}.  
    Cet article propose une analyse approfondie du moteur graphique de *DOOM (2016)*, en explorant les techniques de rendu utilisées,  
    telles que le rendu différé, l'éclairage, la gestion des ombres, le post-traitement et l'optimisation des performances.  
    L'auteur décortique les effets visuels et les choix techniques ayant permis au jeu d'atteindre un haut niveau de fluidité et de qualité visuelle.  
    Consulté le 1 avril 2025.

    \bibitem{valve_source_engine} 
    Valve Corporation, \textit{Source Engine Features}.  
    Disponible en ligne : \url{https://developer.valvesoftware.com/wiki/Source_Engine_Features}.  
    Cette page du wiki officiel de Valve décrit les fonctionnalités du moteur Source, utilisé dans de nombreux jeux  
    comme *Half-Life 2*, *Portal* et *Counter-Strike: Source*. Elle couvre des aspects tels que le rendu graphique,  
    la gestion de la physique, le réseau, l'intelligence artificielle et les outils de développement fournis avec le moteur.  
    Consulté le 1 avril 2025.

    \bibitem{wikipedia_cryengine} 
    Wikipedia, \textit{CryEngine}.  
    Disponible en ligne : \url{https://en.wikipedia.org/wiki/CryEngine}.  
    Cet article de Wikipedia fournit une vue d'ensemble détaillée de CryEngine, un moteur de jeu développé par Crytek.  
    Il explore ses caractéristiques techniques, ses applications dans les jeux vidéo, ses versions successives et ses améliorations au fil du temps.  
    CryEngine est reconnu pour ses capacités de rendu photoréaliste et son utilisation dans des jeux populaires comme *Crysis*.  
    Consulté le 1 avril 2025.

    \bibitem{chanhaeng_normalparallax} 
    Chanhaeng, \textit{Normal/Parallax Mapping with Self-Shadowing}.  
    Disponible en ligne : \url{https://chanhaeng.blogspot.com/2019/01/normalparllax-mapping-with-self.html}.  
    Cet article explique en détail l'implémentation du normal mapping et du parallax mapping avec auto-ombrage.  
    L'auteur y décrit les concepts théoriques de ces techniques graphiques avancées et leur application dans les moteurs de rendu.  
    Le blog fournit également un code exemple pour implémenter ces effets dans les shaders.  
    Consulté le 1 avril 2025.

    \bibitem{silicon_graphics}
    Silicon Graphics.
    Disponible en ligne : \url{https://www.sgi.com/}.
    Silicon Graphics, Inc. (SGI) est une entreprise américaine spécialisée dans la conception de stations de travail et de superordinateurs.
    Pionnier dans le domaine de l'informatique graphique, elle est à l'origine de la création d'OpenGL,
    une API de rendu 2D et 3D largement adoptée dans l'industrie. Leur modèle de pipeline graphique a influencé
    de nombreux moteurs de jeu modernes, ayant fait ses preuves dans l'industrie du jeu vidéo lors de leur collaboration avec
    Nintendo en 1996 pour la création de la Nintendo 64 \cite{copetti_n64}.

    \bibitem{copetti_n64} 
    Fabio Copetti, \textit{Nintendo 64}.  
    Disponible en ligne : \url{https://www.copetti.org/writings/consoles/nintendo-64/}.  
    Cet article offre une analyse détaillée de la console Nintendo 64, y compris son architecture matérielle, ses composants, et les défis techniques rencontrés lors de son développement.  
    L'auteur explore également les innovations apportées par la N64, comme le processeur graphique et le stockage sur cartouche.  
    Consulté le 1 avril 2025.

    \bibitem{lalanguefrancaise}
    La Langue Française.
    Disponible en ligne : \url{https://www.lalanguefrancaise.com/}.
    La Langue Française est un site dédié à la langue française, proposant des ressources, des articles et des outils pour améliorer la maîtrise de la langue.
    Il aborde divers sujets tels que la grammaire, le vocabulaire, l'orthographe et la culture francophone.
    Le site est une référence pour les francophones souhaitant approfondir leur connaissance de la langue française.

    \bibitem{obj_format}
    Wavefront Technologies, \textit{OBJ File Format}.
    Disponible en ligne : \url{https://en.wikipedia.org/wiki/Wavefront_.obj_file}.
    Le format de fichier OBJ est un format de fichier standard pour représenter des objets 3D.
    Il est largement utilisé dans l'industrie de la modélisation 3D et est pris en charge par de nombreux logiciels de modélisation.
    Le format OBJ est simple et facile à lire, ce qui en fait un choix populaire pour l'échange de données 3D entre différentes applications.
    Il permet de stocker des informations sur la géométrie et éventuellement des textures, des matériaux et d'autres attributs associés à un objet 3D s'il est
    accompagné d'un fichier MTL.

    \bibitem{glsl}
    Khronos Group, \textit{OpenGL Shading Language (GLSL)}.
    Disponible en ligne : \url{https://www.khronos.org/opengl/wiki/OpenGL_Shading_Language}.
    OpenGL Shading Language (GLSL) est un langage de
    programmation de shaders de haut niveau dont la syntaxe est fondée sur
    le langage C.
    

\end{thebibliography}