\newpage
    \textbf{Résumé}\\
    Ce projet consiste à développer un jeu vidéo d’horreur en 3D en langage C,
    se déroulant dans l’institut Claude Chappe. Le joueur doit y résoudre des
    mini-jeux tout en échappant à un monstre qui le poursuit. Le jeu repose sur
    un moteur 3D original, entièrement conçu par notre équipe, sans recourir à un
    moteur existant, et intègre modélisation 3D, gameplay et éléments sonores.
    Le projet s’inscrit dans le cadre de la formation de L2 Informatique de
    l’Université du Mans (année universitaire 2024–2025).\\\\
\section{Introduction}
\label{sec:intro}
Ce document présente le projet du jeu Chapper's Fallout réalisé dans le cadre de
la formation de L2 informatique de l’université du Mans pendant la période de
janvier à avril 2025 (mais développé en avance depuis septembre 2024). Ce projet
a été développé en langage C avec la librairie SDL.\\

Dans ce jeu, le joueur peut se déplacer dans l’institut Claude Chappe, avec pour
objectif de résoudre tous les mini-jeux 2D disponibles sur les PC disséminés dans
différents endroits de la zone. Il doit réussir cela tout en survivant au monstre
Chapper, qui le poursuit par moments, et en surmontant des parcours semés de trous
et de plateformes.\\

Le joueur possède un saut pour l’aider dans ces parcours, ainsi qu’une lampe torche
pour éclairer son chemin. Un tutoriel introduit les principes du jeu et les contrôles
permettant les déplacements, l’utilisation de la lampe et du saut avant que la
partie principale ne commence.\\

Nous présenterons dans une première partie notre jeu, ses scénarios d’utilisation
et ses principales fonctionnalités, puis dans une deuxième partie la gestion du
projet. Ensuite, dans une troisième partie, nous détaillerons les éléments principaux
de conception (algorithmes, structures de données, etc.). Nous exposerons par la
suite l’architecture de notre application (structuration du code en fichiers) avant
de montrer les principaux résultats obtenus. Enfin, nous conclurons sur les points
forts et les limites de notre travail, les écarts entre la planification
prévisionnelle et le déroulement réel du projet, ainsi que les leçons tirées de
cette expérience.\\

En annexe, nous présenterons un exemple de débogage et des tests (jeux d’essai
et cas de test d’un exemple au moins — le fichier .c étant disponible dans le
dépôt Git dans un répertoire dédié "test").
\newpage
