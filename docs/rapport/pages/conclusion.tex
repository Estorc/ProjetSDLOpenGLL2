\section{Résultats et conclusion}

Finalement, nous avons réussi à créer un jeu vidéo d'horreur en 3D, avec un moteur
3D original, sans utiliser de moteur existant. Le jeu est fonctionnel et
permet au joueur de se déplacer dans l'institut Claude Chappe, d'interagir avec
des objets, de résoudre des mini-jeux et d'éviter un monstre qui le poursuit.
Nous avons également eu l'occasion d'intégrer des éléments sonores et visuels pour
améliorer l'expérience de jeu.\\

Ce fut un projet ambitieux qui nous a permis de mettre en pratique nos
connaissances en programmation, en modélisation 3D et en conception de jeux vidéo.
Nous avons appris à travailler en équipe, à gérer un projet de A à Z et à surmonter
des défis techniques.\\

Les principales leçons tirées de ce projet sont :\\
\begin{itemize}
    \item Une meilleure gestion du temps et des tâches aurait permis d'éviter le stress
    de dernière minute.
    \item La communication au sein de l'équipe est essentielle pour s'assurer que tout le monde
    est sur la même longueur d'onde et que les tâches sont bien réparties.
    \item La documentation est cruciale pour faciliter la compréhension du code et des
    fonctionnalités du jeu.
    \item La modélisation 3D et l'intégration de ressources dans un moteur de jeu sont des
    étapes complexes qui nécessitent de la patience et de la rigueur.
    \item La gestion des collisions et des interactions entre objets est un aspect clé du
    développement de jeux vidéo, et il est important de bien le planifier dès le début.
\end{itemize}

Quant aux aprentissages, nous avons acquis des compétences en :\\
\begin{itemize}
    \item Programmation en C et utilisation de la bibliothèque SDL pour le développement de jeux.
    \item Modélisation 3D avec Blender et intégration de ressources dans un moteur de jeu.
    \item Gestion de projet et travail en équipe.
    \item Conception d'un moteur de rendu 3D complet et avancé avec OpenGL.
\end{itemize}