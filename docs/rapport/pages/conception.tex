\newpage
\section{Conception}

\subsection{Cahier des charges}

\textbf{- Contexte et objectifs du projet :}\\

Ce projet, réalisé dans le cadre de la L2 Informatique à l’Université du Mans,
consiste à concevoir un jeu vidéo en langage C. Nous avons choisi de créer un jeu
d’horreur en 3D prenant place dans l’institut Claude Chappe où le joueur explore un
bâtiment, parfois poursuivi par un monstre, tout en résolvant des mini-jeux (voir
détails dans l'introduction \ref{sec:intro}).\\

\textbf{- Contraintes techniques :}\\\\
- Langage imposé : C (exclusivement ou presque)  
- Aucune utilisation de moteur de jeu existant (Unity, Unreal…)  
- Bibliothèques autorisées : SDL / OpenGL  
- Durée du projet : janvier à avril 2025 (avec un travail amorcé dès septembre 2024)\\

\textbf{- Fonctionnalités prévues :}\\\\
- Déplacement en 3D dans une reproduction fidèle du bâtiment  
- Monstre poursuivant le joueur à certains moments  
- Mini-jeux en 2D intégrés dans des PC interactifs  
- Mécaniques de plateforme (saut, trous, obstacles)  
- Lampe torche pour éclairer l’environnement  
- Tutoriel expliquant les commandes\\

\textbf{- Architecture générale :}\\\\
Pour cela, nous développons notre propre moteur 3D, \textit{RaptiquaX}, en C avec SDL.
L’institut est modélisé en 3D sur Blender, puis intégré au moteur. Un éditeur de niveau
nous permet de placer objets, collisions et éléments interactifs. Les mini-jeux sont
codés séparément puis intégrés au jeu.\\

\textbf{- Répartition des rôles :}\\\\
- \textbf{Lucien} : modélisation 3D du bâtiment/objets, design sonore\\ 
- \textbf{Loup} : moteur 3D, éditeur de niveau, mécaniques principales\\  
- \textbf{Ekrem} : conception et développement des mini-jeux  \\
- \textbf{Mehdi} : placement des collisions et éléments de jeu via l’éditeur\\

\newpage
\subsection{Fonctionnalités}

Chapper's Fallout est un jeu d’horreur en 3D où le joueur doit explorer l’institut
Claude Chappe tout en évitant un monstre qui le poursuit et en utilisant des
plateformes pour naviguer dans l’environnement. Pour mener à bien ce projet,
nous avons développé un moteur 3D en C, nommé \textit{RaptiquaX}, qui gère les
déplacements, les collisions et les interactions avec l’environnement, en plus
de permettre un affichage 3D avancé avec OpenGL.\\

\begin{itemize}
    \item \emph{Déplacement en 3D} - Le joueur peut se déplacer dans l’institut Claude
    Chappe en utilisant les touches ZQSD (ou flèches directionnelles) pour avancer,
    reculer et tourner. Il peut également sauter avec la touche espace.\\

    \item \emph{Monstre} - Le joueur est poursuivi par un monstre qui apparaît à certains
    moments. Il doit éviter d'être touché par le monstre, sinon il perd une vie.\\

    \item \emph{Mini-jeux} - Le joueur peut interagir avec des ordinateurs pour accéder
    à des mini-jeux en 2D. Ces mini-jeux sont variés et nécessitent différentes
    compétences pour être résolus.\\

    \item \emph{Lampe torche} - Le joueur peut activer une lampe torche pour
    éclairer son chemin. \\

    \item \emph{Vidéo exclusive} - Le joueur peut visionner une vidéo exclusive à la fin
    du jeu, qui lui explique la
    fin de l’histoire.\\

    \item \emph{Affectation des touches} - Le joueur peut configurer les touches de son
    clavier pour contrôler le jeu.\\

    \item \emph{Guide de jeu} - Un tutoriel interactif guide le joueur à travers les
    différentes étapes du jeu, séparés sous forme de chapitres.\\
\end{itemize}
\newpage