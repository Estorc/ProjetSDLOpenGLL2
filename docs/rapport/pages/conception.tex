\newpage
\section{Conception}

\subsection{Cahier des charges}

\textbf{- Contexte et objectifs du projet :}\\

Ce projet, réalisé dans le cadre de la L2 Informatique à l’Université du Mans, consiste à concevoir un jeu vidéo en langage C. Nous avons choisi de créer un jeu d’horreur en 3D prenant place dans l’institut Claude Chappe où le joueur explore un bâtiment, parfois poursuivi par un monstre, tout en résolvant des mini-jeux (voir détails dans l'introduction \ref{sec:intro}).\\

\textbf{- Contraintes techniques :}\\\\
- Langage imposé : C (exclusivement ou presque)  
- Aucune utilisation de moteur de jeu existant (Unity, Unreal…)  
- Bibliothèques autorisées : SDL / OpenGL  
- Durée du projet : janvier à avril 2025 (avec un travail amorcé dès septembre 2024)\\

\textbf{- Fonctionnalités prévues :}\\\\
- Déplacement en 3D dans une reproduction fidèle du bâtiment  
- Monstre poursuivant le joueur à certains moments  
- Mini-jeux en 2D intégrés dans des PC interactifs  
- Mécaniques de plateforme (saut, trous, obstacles)  
- Lampe torche pour éclairer l’environnement  
- Tutoriel expliquant les commandes\\

\textbf{- Architecture générale :}\\\\
Pour cela, nous développons notre propre moteur 3D, \textit{RaptiquaX}, en C avec SDL. L’institut est modélisé en 3D sur Blender, puis intégré au moteur. Un éditeur de niveau nous permet de placer objets, collisions et éléments interactifs. Les mini-jeux sont codés séparément puis intégrés au jeu.\\

\textbf{- Répartition des rôles :}\\\\
- \textbf{Lucien} : modélisation 3D du bâtiment/objets, design sonore\\ 
- \textbf{Loup} : moteur 3D, éditeur de niveau, mécaniques principales\\  
- \textbf{Ekrem} : conception et développement des mini-jeux  \\
- \textbf{Mehdi} : placement des collisions et éléments de jeu via l’éditeur\\

\newpage
\subsection{Fonctionnalités}
\newpage