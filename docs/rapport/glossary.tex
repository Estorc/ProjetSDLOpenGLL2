\setabbreviationstyle{long-short}

\newglossaryentry{openGL}
{
    name=openGL,
    description={Branche libre de la librairie graphique GL développée par
    Silicon Graphics en 1992 et reprise par le groupe Khronos à partir de
    2006}
}

\newglossaryentry{GLSL}
{
    name=GLSL,
    description={OpenGL Shading Language (GLSL) est un langage de
    programmation de shaders de haut niveau dont la syntaxe est fondée sur
    le langage C.}
}

\newglossaryentry{SDL2}
{
    name=SDL2,
    description={Simple DirectMedia Layer 2 (SDL2) est la deuxième version de
    SDL, une librairie graphique développée par Sam Lantinga ; acclamée pour
    sa portabilité et son accessibilité}
}

\newglossaryentry{vertex shader}
{
    name=vertex shader,
    description={Le \gls{vertex} shader est responsable du traitement des
    primitives (triangles, lignes, points) lors de la pipeline graphique}
}

\newglossaryentry{vertex}
{
    name=vertex,
    description={Un vertex est un point dans l'espace 3D qui définit la
    position d'un sommet d'une primitive (triangle, ligne, point) dans la
    pipeline graphique}
}

\newglossaryentry{fragment shader}
{
    name=fragment shader,
    description={Le fragment shader est responsable du traitement des
    fragments lors de la pipeline graphique}
}

\newglossaryentry{fragment}
{
    name=fragment,
    description={Un fragment est un composant graphique composé de peu de
    pixels qui constitue une primitive en informatique graphique}
}

\newglossaryentry{shader}
{
    name=shader,
    description={Un shader est un programme graphique compilé en microcode
    graphique sur le \gls{gpu}}
}

\newglossaryentry{rasteriser}
{
    name=rasteriser,
    description={(Informatique) Convertir des données numériques en points
    permettant leur impression ou leur affichage. \cite{lalanguefrancaise}}
}

\newglossaryentry{GBuffer}
{
    name=G-Buffer,
    description={Le GBuffer est un tampon de rendu utilisé dans le rendu
    différé pour stocker les informations de la scène, telles que la couleur,
    les normales et la profondeur}
}

\newglossaryentry{shadow mapping}
{
    name=shadow mapping,
    description={Le shadow mapping est une technique de rendu d'ombres
    utilisée dans le rendu différé. Elle consiste à créer une carte d'ombres
    (shadow map) qui stocke les informations de profondeur de la scène
    depuis la perspective de la source lumineuse. Lors du rendu, on
    compare la profondeur d'un fragment avec la profondeur stockée dans la
    shadow map pour déterminer s'il est dans l'ombre ou non}
}

\newglossaryentry{ssr}
{
    name=SSR,
    description={Le SSR (Screen Space Reflection) est une technique de rendu
    utilisée pour créer des réflexions réalistes en utilisant les
    informations de la scène stockées dans le \gls{GBuffer}. Elle permet de
    simuler des réflexions sur des surfaces réfléchissantes en se basant
    sur les pixels visibles à l'écran},
    text={SSR},
    first={SSR (Screen Space Reflection)}
}

\newglossaryentry{ssao}
{
    name=SSAO,
    description={L'SSAO (Screen Space Ambient Occlusion) est une technique de
    rendu utilisée pour simuler l'occlusion ambiante dans les scènes 3D. Elle
    permet de créer des ombres douces et réalistes en tenant compte de la
    géométrie de la scène et de la position des lumières},
    text={SSAO},
    first={SSAO (Screen Space Ambient Occlusion)}
}
\newglossaryentry{bloom}
{
    name=bloom,
    description={Le bloom est une technique de post-traitement utilisée pour
    simuler l'effet de diffusion de la lumière sur les surfaces brillantes.
    Elle crée un halo lumineux autour des objets lumineux pour améliorer le
    réalisme de la scène}
}
\newglossaryentry{smaa}
{
    name=SMAA,
    description={Le SMAA (Subpixel Morphological Anti-Aliasing) est une
    technique d'anti-aliasing utilisée pour réduire les artefacts de
    crénelage (souvent appelé \emph{effet escalier}) dans les images 3D. Elle combine plusieurs techniques
    d'anti-aliasing pour obtenir un rendu plus lisse et réaliste},
    text={SMAA},
    first={SMAA (Subpixel Morphological Anti-Aliasing)}
}

\newacronym{gpu}{GPU}{processeur graphique}